\section{Princip rada analognih računara}
Analogni računari vrše izračunavanja tako sto obrađuju kontinualne veličine. Iako se  za njihov rad najčešće koristi električna struja i napon, to nije nikakvo pravilo, i bilo koja kontinualna veličina može da se upotrebi. Te veličine su na primer: \begin{itemize}
				\item Količina vode u cevima
				\item Zategnutost opruga
				\item Intenzitet magnetne sile
				\item Vibracije tla kod seizmografa
				\item Visina vode kod mašine za predvidjanje plime i oseke \cite{tide}
				\item Položaj zubčanika
				\item Položaj čekrka
				\item Temperatura raznih supstanci
			\end{itemize}
			
\bigskip

    \begin{primer}
    Uzmimo kao primer operaciju sabiranja dva borja koristeći količinu vode u posudama.\\
    Da bismo sabrali dva broja, recimo 3 i 5, poterebno je da imamo tri identične čaše sa skalom za merenje nivoa vode u čaši. Sabiranje možemo izvrsiti prateći sledeći postupak:\begin{enumerate}
        \item U prvu čašu nalijemo vode do trećeg podeoka (za broj 3).
        \item U drugu čašu nalijemo vodu do petog podeoka (za broj 5).
        \item Prelijemo sadržaj obe čaše u treću času koju smo ostavili praznu.
        \item Očitamo visinu vode tako što gledamo do kog podeoka je stigla.
        \item Uočavanjem da je visina vode stigla do osmog podeoka (za broj 8) dobili smo rezultat sabiranja.
    \end{enumerate}
    \end{primer}
    
\bigskip

    \begin{primer}
    Za množenje dva broja možemo koristiti električnu struju, napon i otpor.\\
    Da bismo pomnožili dva broja, recimo 3 i 5, poterebno je da imamo voltmetar i električno kolo koje se sastoji od strujnog generatora i promenljivog otpornika. Množenje možemo izvrsiti prateći sledeći postupak:\begin{enumerate}
        \item Podesimo strujni generator tako da generiše električnu struju od tri ampera (za broj 3).
        \item Podesimo promenljivi otpornik tako da mu otpor bude pet oma (za broj 5).
        \item Postavimo pipalice voltmetra na krajeve otpornika.
        \item Očitamo napon na voltmetru.
        \item Uočavanjem da je napon očitan na volmetru jednak petnaest volti (za broj 15) dobili smo rezultat množenja.
    \end{enumerate}
    \centering Ovaj postupak koristi formulu
    \centering $$ U = R*I $$\\
    \centering za računanje proizvoda.
    \end{primer}

